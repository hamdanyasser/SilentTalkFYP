% ============================================================
% SilentTalk FYP - Partnership Collaboration Guide
% Team: Yasser (Backend/Infrastructure) + Zainab (Frontend/UI)
% Timeline: January - July 2026
% ============================================================

\documentclass[12pt,a4paper]{article}

% Packages
\usepackage[utf8]{inputenc}
\usepackage[T1]{fontenc}
\usepackage[english]{babel}
\usepackage{geometry}
\usepackage{fancyhdr}
\usepackage{graphicx}
\usepackage{xcolor}
\usepackage{listings}
\usepackage{enumitem}
\usepackage{longtable}
\usepackage{booktabs}
\usepackage{tabularx}
\usepackage{hyperref}
\usepackage{tikz}
\usepackage{amssymb}
\usepackage{amsmath}
\usepackage{tcolorbox}
\usepackage{multirow}

% Page setup
\geometry{
    left=2.5cm,
    right=2.5cm,
    top=3cm,
    bottom=3cm
}

% Colors
\definecolor{primaryblue}{RGB}{41, 128, 185}
\definecolor{secondarygreen}{RGB}{39, 174, 96}
\definecolor{warningorange}{RGB}{230, 126, 34}
\definecolor{dangerred}{RGB}{231, 76, 60}
\definecolor{codebg}{RGB}{245, 245, 245}

% Hyperref setup
\hypersetup{
    colorlinks=true,
    linkcolor=primaryblue,
    filecolor=magenta,
    urlcolor=primaryblue,
    pdftitle={SilentTalk FYP Partnership Guide},
    pdfauthor={Yasser \& Zainab},
}

% Code listing setup
\lstset{
    basicstyle=\ttfamily\small,
    backgroundcolor=\color{codebg},
    frame=single,
    breaklines=true,
    captionpos=b,
    numbers=left,
    numberstyle=\tiny\color{gray},
    keywordstyle=\color{primaryblue}\bfseries,
    commentstyle=\color{secondarygreen},
    stringstyle=\color{dangerred},
    showstringspaces=false,
    tabsize=2
}

% Custom commands
\newcommand{\checkbox}{\Square}
\newcommand{\checkedbox}{\CheckedBox}
\newcommand{\priority}[1]{\textcolor{dangerred}{\textbf{#1}}}
\newcommand{\status}[1]{\textcolor{secondarygreen}{\textbf{#1}}}
\newcommand{\warning}[1]{\textcolor{warningorange}{\textbf{#1}}}

% Headers and footers
\pagestyle{fancy}
\fancyhf{}
\fancyhead[L]{SilentTalk FYP - Partnership Guide}
\fancyhead[R]{\thepage}
\fancyfoot[C]{January - July 2026}
\renewcommand{\headrulewidth}{0.4pt}
\renewcommand{\footrulewidth}{0.4pt}

% Title page
\title{
    \Huge\textbf{SilentTalk FYP}\\
    \huge Partnership Collaboration Guide\\
    \vspace{0.5cm}
    \Large Team Success Roadmap
}
\author{
    \textbf{Yasser} - Backend \& Infrastructure Lead\\
    \textbf{Zainab} - Frontend \& UI/UX Lead\\
    \vspace{0.3cm}\\
    Final Year Project\\
    January - July 2026
}
\date{\today}

% Document begins
\begin{document}

\maketitle
\thispagestyle{empty}

\newpage
\tableofcontents
\newpage

% ============================================================
\section{Welcome \& Introduction}
% ============================================================

\subsection{Purpose of This Guide}

This comprehensive collaboration guide is designed to ensure seamless teamwork between Yasser (backend/infrastructure) and Zainab (frontend/UI) throughout the SilentTalk FYP project. The guide provides:

\begin{itemize}[leftmargin=*]
    \item \textbf{Complete setup instructions} for Zainab to get started
    \item \textbf{Accurate project status} based on current codebase analysis
    \item \textbf{Clear task assignments} with realistic time estimates
    \item \textbf{Week-by-week roadmap} from January to July 2026
    \item \textbf{Technical references} and troubleshooting guides
    \item \textbf{Collaboration workflows} for efficient teamwork
\end{itemize}

\subsection{Project Vision}

\textit{``Breaking down communication barriers and creating an inclusive environment for the deaf and hard-of-hearing community through advanced technology.''}

\vspace{0.5cm}

SilentTalk is a comprehensive sign language communication platform that combines:
\begin{itemize}
    \item Real-time sign language recognition using machine learning
    \item Accessible video conferencing with live captions
    \item Community features (forums, resources, glossary)
    \item Professional interpreter booking services
\end{itemize}

\subsection{Team Roles \& Responsibilities}

\begin{table}[h]
\centering
\begin{tabularx}{\textwidth}{|l|X|}
\hline
\rowcolor{codebg}
\textbf{Role} & \textbf{Responsibilities} \\ \hline
\textbf{Yasser} & Backend API development, Database design, Infrastructure setup, Docker orchestration, SignalR implementation, ML service integration, Security, Deployment \\ \hline
\textbf{Zainab} & Frontend development, UI/UX design, Component library, React integration, Accessibility implementation, User testing, Documentation \\ \hline
\textbf{Shared} & Code reviews, Weekly planning, Testing, Documentation, Timeline management, Quality assurance \\ \hline
\end{tabularx}
\end{table}

\subsection{Timeline Overview}

\begin{tcolorbox}[colback=secondarygreen!10, colframe=secondarygreen, title=7-Month Timeline]
\textbf{January 2026:} Setup \& Planning\\
\textbf{February:} Core authentication \& video calling\\
\textbf{March:} ML integration \& caption system\\
\textbf{April:} User management \& contacts\\
\textbf{May:} Community features\\
\textbf{June:} Testing \& polish\\
\textbf{July:} Final testing \& submission
\end{tcolorbox}

% ============================================================
\section{Setup Guide for Zainab}
% ============================================================

\subsection{Prerequisites}

Before starting, ensure you have the following installed on your machine:

\begin{enumerate}[leftmargin=*]
    \item \textbf{Git} (version 2.30+)
    \begin{lstlisting}[language=bash]
# Verify installation
git --version
    \end{lstlisting}

    \item \textbf{Docker Desktop} (latest version)
    \begin{lstlisting}[language=bash]
# Verify installation
docker --version
docker-compose --version
    \end{lstlisting}

    \item \textbf{Node.js} (v20.x LTS)
    \begin{lstlisting}[language=bash]
# Verify installation
node --version  # Should be v20.x
npm --version   # Should be 10.x
    \end{lstlisting}

    \item \textbf{Visual Studio Code} (recommended IDE)

    \item \textbf{Terminal/Command Prompt}
\end{enumerate}

\subsection{Step-by-Step Setup}

\subsubsection{Step 1: Clone Repository}

\begin{lstlisting}[language=bash]
# Clone the repository
git clone https://github.com/hamdanyasser/SilentTalkFYP.git

# Navigate to project directory
cd SilentTalkFYP

# Check current branch
git branch
# You should be on: main
\end{lstlisting}

\subsubsection{Step 2: Install Frontend Dependencies}

\begin{lstlisting}[language=bash]
# Navigate to client directory
cd client

# Install Node.js dependencies
npm install

# This will install:
# - React 18
# - TypeScript
# - Vite
# - All required libraries
\end{lstlisting}

\warning{Note:} The project uses \texttt{npm install} (not \texttt{npm ci}) because there's no \texttt{package-lock.json}.

\subsubsection{Step 3: Start Docker Services}

\begin{lstlisting}[language=bash]
# Go back to project root
cd ~/SilentTalkFYP

# Start all services with one command
./start.sh

# Wait ~60 seconds for services to initialize
# You'll see status updates in the terminal
\end{lstlisting}

This script starts:
\begin{itemize}
    \item PostgreSQL (database)
    \item MongoDB (messages/logs)
    \item Redis (caching)
    \item MinIO (file storage)
    \item Backend API (ASP.NET Core)
    \item ML Service (FastAPI)
    \item Frontend (React + Vite)
\end{itemize}

\subsubsection{Step 4: Verify Services Are Running}

\begin{lstlisting}[language=bash]
# Check service status
docker ps

# You should see 7-8 containers running:
# - silents-talk-postgres
# - silents-talk-mongodb
# - silents-talk-redis
# - silents-talk-minio
# - silents-talk-server
# - silents-talk-ml
# - silents-talk-client
\end{lstlisting}

\subsubsection{Step 5: Access the Application}

Open your browser and navigate to:

\begin{table}[h]
\centering
\begin{tabular}{|l|l|}
\hline
\rowcolor{codebg}
\textbf{Service} & \textbf{URL} \\ \hline
Frontend (your work!) & \url{http://localhost:3000} \\ \hline
Backend API Docs & \url{http://localhost:5000/docs} \\ \hline
ML Service API & \url{http://localhost:8000/docs} \\ \hline
MinIO Console & \url{http://localhost:9001} \\ \hline
\end{tabular}
\end{table}

\subsection{VS Code Setup}

\subsubsection{Recommended Extensions}

Install these VS Code extensions for optimal development experience:

\begin{enumerate}
    \item \textbf{ESLint} - JavaScript/TypeScript linting
    \item \textbf{Prettier} - Code formatting
    \item \textbf{ES7+ React Snippets} - React code snippets
    \item \textbf{Auto Rename Tag} - HTML/JSX tag renaming
    \item \textbf{GitLens} - Git integration
    \item \textbf{Thunder Client} - API testing (alternative to Postman)
    \item \textbf{Error Lens} - Inline error display
    \item \textbf{Path Intellisense} - File path autocomplete
\end{enumerate}

\subsubsection{VS Code Settings}

Create \texttt{.vscode/settings.json} in project root:

\begin{lstlisting}[language=json]
{
  "editor.formatOnSave": true,
  "editor.defaultFormatter": "esbenp.prettier-vscode",
  "editor.codeActionsOnSave": {
    "source.fixAll.eslint": true
  },
  "typescript.tsdk": "node_modules/typescript/lib",
  "files.exclude": {
    "**/node_modules": true,
    "**/dist": true
  }
}
\end{lstlisting}

\subsection{Troubleshooting Common Issues}

\subsubsection{Docker Won't Start}

\begin{tcolorbox}[colback=dangerred!10, colframe=dangerred, title=Problem: Docker containers fail to start]
\textbf{Solution:}
\begin{lstlisting}[language=bash]
# Stop all containers
cd ~/SilentTalkFYP
./stop.sh

# Clean Docker system
docker system prune -f

# Restart Docker Desktop
# Then try starting again
./start.sh
\end{lstlisting}
\end{tcolorbox}

\subsubsection{Port Already in Use}

\begin{tcolorbox}[colback=warningorange!10, colframe=warningorange, title=Problem: Port 3000/5000/8000 already in use]
\textbf{Solution:}
\begin{lstlisting}[language=bash]
# Find what's using the port (Linux/Mac)
lsof -i :3000
lsof -i :5000

# Kill the process
kill -9 <PID>

# Or change ports in docker-compose.yml if needed
\end{lstlisting}
\end{tcolorbox}

\subsubsection{Frontend Won't Build}

\begin{tcolorbox}[colback=warningorange!10, colframe=warningorange, title=Problem: npm install fails or build errors]
\textbf{Solution:}
\begin{lstlisting}[language=bash]
cd ~/SilentTalkFYP/client

# Clear cache and reinstall
rm -rf node_modules
rm package-lock.json
npm cache clean --force
npm install

# If still issues, check Node version
node --version  # Must be v20.x
\end{lstlisting}
\end{tcolorbox}

\subsection{Verify Setup Checklist}

Complete this checklist to ensure everything is working:

\begin{itemize}
    \item[$\square$] Git repository cloned successfully
    \item[$\square$] Docker Desktop is running
    \item[$\square$] All 7-8 Docker containers are running (check with \texttt{docker ps})
    \item[$\square$] Frontend accessible at \url{http://localhost:3000}
    \item[$\square$] Backend API docs at \url{http://localhost:5000/docs}
    \item[$\square$] Can register a test user successfully
    \item[$\square$] Can login with test user
    \item[$\square$] VS Code opens project without errors
    \item[$\square$] Can run \texttt{npm run dev} in client directory
\end{itemize}

Once all items are checked, you're ready to start development! 🎉

% ============================================================
\section{Project Status (Current State)}
% ============================================================

\subsection{Overall Completion: \texorpdfstring{\textasciitilde}{}20\%}

Based on thorough codebase analysis, here's the accurate current status:

\begin{table}[h]
\centering
\begin{tabularx}{\textwidth}{|l|c|X|}
\hline
\rowcolor{codebg}
\textbf{Component} & \textbf{Status} & \textbf{Notes} \\ \hline
Infrastructure & \status{95\%} & Docker, databases, all services running \\ \hline
Database & \status{90\%} & PostgreSQL + MongoDB configured \\ \hline
Backend Auth & \status{60\%} & JWT working, email verification pending \\ \hline
Backend Calls & \status{80\%} & Full call management + SignalR hub \\ \hline
ML Service & \warning{30\%} & Demo mode, no trained model \\ \hline
Frontend Auth & \status{70\%} & Login/register pages working \\ \hline
Frontend Video & \warning{40\%} & UI exists, WebRTC not connected \\ \hline
Frontend Other & \warning{15\%} & Basic pages, needs backend integration \\ \hline
Testing & \priority{10\%} & Minimal tests exist \\ \hline
\end{tabularx}
\end{table}

\subsection{What's Working RIGHT NOW}

\begin{tcolorbox}[colback=secondarygreen!10, colframe=secondarygreen, title=✅ Fully Functional Features]
\begin{enumerate}
    \item \textbf{Infrastructure} - All Docker services running, data persists
    \item \textbf{Authentication} - Users can register/login, JWT tokens work
    \item \textbf{Call Management Backend} - Full CRUD for calls, scheduling, history
    \item \textbf{SignalR Hub} - Complete WebRTC signaling (601 lines of code!)
    \item \textbf{Recording Upload} - Can upload/download call recordings to MinIO
    \item \textbf{Admin Panel} - User management, statistics, audit logs
    \item \textbf{Frontend Login/Register} - Forms work, connected to backend
    \item \textbf{Video Call Page} - UI exists with caption overlay
    \item \textbf{ML Service Connection} - WebSocket streaming ready (demo mode)
    \item \textbf{Design System} - Button, Input, Modal components ready
\end{enumerate}
\end{tcolorbox}

\subsection{What's Missing (Needs Implementation)}

\begin{tcolorbox}[colback=dangerred!10, colframe=dangerred, title=❌ Critical Missing Features]
\begin{enumerate}
    \item \textbf{Trained ML Model} - Currently using mock predictions
    \item \textbf{WebRTC Video Integration} - Backend ready, frontend needs work
    \item \textbf{Contact Management} - Backend controller missing
    \item \textbf{Forum Feature} - Backend not implemented
    \item \textbf{Resource Library} - Backend not implemented
    \item \textbf{Glossary} - Backend not implemented
    \item \textbf{Profile Management} - Backend partially done
    \item \textbf{Frontend UI Pages} - Many pages are placeholders
    \item \textbf{Comprehensive Testing} - Only 10\% coverage
    \item \textbf{Email Sending} - No SMTP configured
\end{enumerate}
\end{tcolorbox}

\subsection{Technology Stack Overview}

\subsubsection{Backend (Yasser's Domain)}

\begin{itemize}
    \item \textbf{Framework:} ASP.NET Core 8.0
    \item \textbf{Language:} C\# 12
    \item \textbf{Database:} PostgreSQL (primary), MongoDB (messages)
    \item \textbf{Caching:} Redis
    \item \textbf{Storage:} MinIO (S3-compatible)
    \item \textbf{Real-time:} SignalR (WebSocket-based)
    \item \textbf{ORM:} Entity Framework Core 8.0
    \item \textbf{Auth:} ASP.NET Core Identity + JWT
    \item \textbf{API Docs:} Swagger/OpenAPI
\end{itemize}

\subsubsection{Frontend (Zainab's Domain)}

\begin{itemize}
    \item \textbf{Framework:} React 18
    \item \textbf{Language:} TypeScript 5.x
    \item \textbf{Build Tool:} Vite
    \item \textbf{State Management:} React Context API
    \item \textbf{Styling:} CSS Modules + Sass
    \item \textbf{HTTP Client:} Axios
    \item \textbf{WebRTC:} simple-peer library
    \item \textbf{SignalR Client:} @microsoft/signalr
    \item \textbf{Testing:} Jest + React Testing Library
\end{itemize}

\subsubsection{ML Service (Shared Integration)}

\begin{itemize}
    \item \textbf{Framework:} FastAPI (Python)
    \item \textbf{ML:} TensorFlow + ONNX Runtime
    \item \textbf{Computer Vision:} MediaPipe
    \item \textbf{Streaming:} WebSocket
    \item \textbf{Status:} Demo mode (mock predictions)
\end{itemize}

\subsection{Repository Structure}

\begin{lstlisting}[language=bash]
SilentTalkFYP/
├── server/                 # Backend API (Yasser)
│   ├── src/
│   │   ├── SilentTalk.Api/
│   │   │   ├── Controllers/     # API endpoints
│   │   │   ├── Hubs/            # SignalR hub
│   │   │   ├── Program.cs       # App entry point
│   │   └── SilentTalk.Domain/   # Entities/models
│   └── database/migrations/
├── client/                 # Frontend (Zainab)
│   ├── src/
│   │   ├── pages/         # Page components
│   │   ├── components/    # Reusable components
│   │   ├── services/      # API clients
│   │   ├── contexts/      # React contexts
│   │   ├── design-system/ # Design components
│   │   ├── hooks/         # Custom hooks
│   │   └── types/         # TypeScript types
│   ├── package.json
│   └── vite.config.ts
├── ml-service/            # ML Service (Integration)
│   ├── app/
│   │   ├── api/          # FastAPI endpoints
│   │   ├── services/     # ML inference
│   │   └── main.py       # FastAPI app
│   └── requirements.txt
├── infrastructure/        # Docker configs
│   └── docker/
│       └── docker-compose.yml
├── docs/                  # Documentation
├── start.sh              # Start everything
├── stop.sh               # Stop everything
└── PROJECT_STATUS.md     # Detailed status
\end{lstlisting}

% ============================================================
\section{Task Assignments \& Division of Labor}
% ============================================================

\subsection{Guiding Principles}

\begin{enumerate}
    \item \textbf{Clear Boundaries:} Yasser owns backend, Zainab owns frontend
    \item \textbf{Communication:} Daily updates on blockers, weekly planning
    \item \textbf{Code Reviews:} Review each other's PRs within 24 hours
    \item \textbf{Shared Goals:} Both responsible for project success
    \item \textbf{Flexibility:} Help each other when needed
\end{enumerate}

\subsection{Yasser's Tasks (Backend/Infrastructure)}

\begin{longtable}{|p{0.5cm}|p{6cm}|p{2cm}|p{2cm}|p{2cm}|}
\hline
\rowcolor{codebg}
\textbf{\#} & \textbf{Task} & \textbf{Priority} & \textbf{Time} & \textbf{Status} \\ \hline
\endfirsthead
\hline
\rowcolor{codebg}
\textbf{\#} & \textbf{Task} & \textbf{Priority} & \textbf{Time} & \textbf{Status} \\ \hline
\endhead

Y1 & Complete Contact Management Backend & High & 1 week & Todo \\ \hline
Y2 & Implement Email Service (SMTP) & High & 3 days & Todo \\ \hline
Y3 & Connect Profile Endpoints to DB & Medium & 2 days & Todo \\ \hline
Y4 & Forum Backend Implementation & Medium & 2 weeks & Todo \\ \hline
Y5 & Resource Library Backend & Medium & 1.5 weeks & Todo \\ \hline
Y6 & Glossary Backend API & Low & 1 week & Todo \\ \hline
Y7 & ML Model Training Setup & Critical & 2 weeks & Todo \\ \hline
Y8 & Backend Unit Tests (80\% coverage) & High & 3 weeks & Todo \\ \hline
Y9 & API Performance Optimization & Medium & 1 week & Todo \\ \hline
Y10 & Security Audit \& Fixes & High & 1 week & Todo \\ \hline
Y11 & Deployment Scripts \& CI/CD & Medium & 1 week & Todo \\ \hline
Y12 & Documentation Updates & Low & Ongoing & Todo \\ \hline
\caption{Yasser's Task List}
\end{longtable}

\subsection{Zainab's Tasks (Frontend/UI)}

\begin{longtable}{|p{0.5cm}|p{6cm}|p{2cm}|p{2cm}|p{2cm}|}
\hline
\rowcolor{codebg}
\textbf{\#} & \textbf{Task} & \textbf{Priority} & \textbf{Time} & \textbf{Status} \\ \hline
\endfirsthead
\hline
\rowcolor{codebg}
\textbf{\#} & \textbf{Task} & \textbf{Priority} & \textbf{Time} & \textbf{Status} \\ \hline
\endhead

Z1 & Complete WebRTC Video Integration & Critical & 2 weeks & Todo \\ \hline
Z2 & Connect Call History Page to Backend & High & 3 days & Todo \\ \hline
Z3 & Profile Page Backend Integration & High & 4 days & Todo \\ \hline
Z4 & Contact Management UI & High & 1 week & Todo \\ \hline
Z5 & Call Scheduling UI & High & 1 week & Todo \\ \hline
Z6 & Forum Frontend Implementation & Medium & 2 weeks & Todo \\ \hline
Z7 & Resource Library Frontend & Medium & 1.5 weeks & Todo \\ \hline
Z8 & Glossary Frontend & Low & 1 week & Todo \\ \hline
Z9 & Responsive Design Polish & High & 1 week & Todo \\ \hline
Z10 & Accessibility Testing (WCAG 2.1) & High & 1 week & Todo \\ \hline
Z11 & Frontend Unit Tests & Medium & 2 weeks & Todo \\ \hline
Z12 & User Documentation & Medium & 1 week & Todo \\ \hline
Z13 & UI/UX Polish & Medium & Ongoing & Todo \\ \hline
\caption{Zainab's Task List}
\end{longtable}

\subsection{Shared Tasks (Collaboration Required)}

\begin{table}[h]
\centering
\begin{tabularx}{\textwidth}{|p{0.5cm}|X|p{2.5cm}|p{2.5cm}|}
\hline
\rowcolor{codebg}
\textbf{\#} & \textbf{Task} & \textbf{Who Leads} & \textbf{Time} \\ \hline
S1 & ML Model Integration & Yasser (backend), Zainab (UI) & 1 week \\ \hline
S2 & End-to-End Testing & Both & 2 weeks \\ \hline
S3 & Performance Testing & Yasser (API), Zainab (UI) & 1 week \\ \hline
S4 & Security Testing & Both & 3 days \\ \hline
S5 & User Acceptance Testing & Both & 1 week \\ \hline
S6 & Bug Fixing Sprint & Both & 2 weeks \\ \hline
S7 & Final Documentation & Both & 1 week \\ \hline
S8 & Deployment & Yasser (infra), Zainab (test) & 3 days \\ \hline
\end{tabularx}
\end{table}

\subsection{Dependencies Matrix}

\textit{Understanding task dependencies prevents blockers:}

\begin{table}[h]
\centering
\small
\begin{tabularx}{\textwidth}{|X|X|}
\hline
\rowcolor{codebg}
\textbf{Frontend Task} & \textbf{Depends On (Backend)} \\ \hline
WebRTC Video (Z1) & SignalR Hub ✅ Already done! \\ \hline
Call History (Z2) & Call endpoints ✅ Ready! \\ \hline
Profile Page (Z3) & User endpoints ⚠️ Yasser: Y3 \\ \hline
Contact UI (Z4) & Contact backend ❌ Yasser: Y1 \\ \hline
Forum UI (Z6) & Forum backend ❌ Yasser: Y4 \\ \hline
Resource Library (Z7) & Resource backend ❌ Yasser: Y5 \\ \hline
Glossary (Z8) & Glossary backend ❌ Yasser: Y6 \\ \hline
\end{tabularx}
\end{table}

\warning{Key Insight:} Zainab can start Z1 (WebRTC) and Z2 (Call History) immediately because backend is ready!

% ============================================================
\section{Implementation Roadmap (January - July 2026)}
% ============================================================

\subsection{Month-by-Month Plan}

\subsubsection{January 2026: Setup \& Foundation}

\textbf{Goals:}
\begin{itemize}
    \item Zainab completes environment setup
    \item Both team members understand codebase
    \item Weekly meeting cadence established
\end{itemize}

\textbf{Deliverables:}
\begin{itemize}
    \item[$\square$] Zainab can run project locally
    \item[$\square$] Git workflow established
    \item[$\square$] Weekly meeting schedule set
    \item[$\square$] Communication channels confirmed
\end{itemize}

\textbf{Tasks (Week by Week):}

\begin{table}[h]
\centering
\begin{tabularx}{\textwidth}{|l|X|}
\hline
\rowcolor{codebg}
\textbf{Week} & \textbf{Activities} \\ \hline
Week 1 (Jan 6-12) & \textbf{Zainab:} Environment setup, run project locally, explore codebase\\
& \textbf{Yasser:} Code walkthrough for Zainab, document current APIs \\ \hline
Week 2 (Jan 13-19) & \textbf{Zainab:} Study existing components, review auth flow\\
& \textbf{Yasser:} Start Y1 (Contact backend) \\ \hline
Week 3 (Jan 20-26) & \textbf{Zainab:} Start Z2 (Call history page)\\
& \textbf{Yasser:} Continue Y1 \\ \hline
Week 4 (Jan 27-31) & \textbf{Both:} First mini-sprint review, plan February \\ \hline
\end{tabularx}
\end{table}

\subsubsection{February 2026: Core Features}

\textbf{Goals:}
\begin{itemize}
    \item WebRTC video calling working end-to-end
    \item Contact management complete
    \item Call history integrated
\end{itemize}

\textbf{Deliverables:}
\begin{itemize}
    \item[$\square$] Users can make 1-on-1 video calls
    \item[$\square$] Contact list shows online/offline status
    \item[$\square$] Call history displays past calls with filters
    \item[$\square$] Profile page connected to backend
\end{itemize}

\textbf{Tasks (Week by Week):}

\begin{table}[h]
\centering
\begin{tabularx}{\textwidth}{|l|X|}
\hline
\rowcolor{codebg}
\textbf{Week} & \textbf{Activities} \\ \hline
Week 5 (Feb 3-9) & \textbf{Zainab:} Start Z1 (WebRTC integration - Part 1)\\
& \textbf{Yasser:} Finish Y1 (Contact backend), start Y3 (Profile DB) \\ \hline
Week 6 (Feb 10-16) & \textbf{Zainab:} Z1 (WebRTC - Part 2), Z2 complete\\
& \textbf{Yasser:} Y3 complete, start Y2 (Email service) \\ \hline
Week 7 (Feb 17-23) & \textbf{Zainab:} Z3 (Profile page), Z4 start (Contact UI)\\
& \textbf{Yasser:} Y2 complete, review Z1 \\ \hline
Week 8 (Feb 24-28) & \textbf{Both:} Integration testing video calls, sprint review \\ \hline
\end{tabularx}
\end{table}

\subsubsection{March 2026: ML Integration \& Captions}

\textbf{Goals:}
\begin{itemize}
    \item ML model trained and integrated
    \item Real-time captions working in video calls
    \item Call scheduling implemented
\end{itemize}

\textbf{Deliverables:}
\begin{itemize}
    \item[$\square$] Sign language recognition (85\%+ accuracy)
    \item[$\square$] Live captions during calls
    \item[$\square$] Call scheduling calendar
    \item[$\square$] Contact management complete
\end{itemize}

\textbf{Tasks:}
\begin{itemize}
    \item \textbf{Yasser:} Y7 (ML training), S1 (ML integration)
    \item \textbf{Zainab:} Z4 complete, Z5 (Scheduling UI), S1 (ML UI)
\end{itemize}

\subsubsection{April 2026: Community Features}

\textbf{Goals:}
\begin{itemize}
    \item Forum operational
    \item Resource library functional
    \item Glossary searchable
\end{itemize}

\textbf{Deliverables:}
\begin{itemize}
    \item[$\square$] Users can create/reply to forum posts
    \item[$\square$] Resource library with videos/tutorials
    \item[$\square$] Searchable sign language glossary
\end{itemize}

\textbf{Tasks:}
\begin{itemize}
    \item \textbf{Yasser:} Y4 (Forum backend), Y5 (Resources), Y6 (Glossary)
    \item \textbf{Zainab:} Z6 (Forum UI), Z7 (Resources UI), Z8 (Glossary UI)
\end{itemize}

\subsubsection{May 2026: Polish \& Responsive Design}

\textbf{Goals:}
\begin{itemize}
    \item All features responsive on mobile/tablet
    \item Accessibility compliance (WCAG 2.1 AA)
    \item UI/UX polished
\end{itemize}

\textbf{Deliverables:}
\begin{itemize}
    \item[$\square$] Mobile-responsive design
    \item[$\square$] Accessibility audit passed
    \item[$\square$] Performance optimized
\end{itemize}

\textbf{Tasks:}
\begin{itemize}
    \item \textbf{Yasser:} Y9 (Performance), Y10 (Security audit)
    \item \textbf{Zainab:} Z9 (Responsive), Z10 (Accessibility), Z13 (UI polish)
\end{itemize}

\subsubsection{June 2026: Testing \& Bug Fixes}

\textbf{Goals:}
\begin{itemize}
    \item Comprehensive testing complete
    \item All critical bugs fixed
    \item 80\%+ code coverage
\end{itemize}

\textbf{Deliverables:}
\begin{itemize}
    \item[$\square$] Unit tests (80\%+ coverage)
    \item[$\square$] Integration tests
    \item[$\square$] End-to-end tests
    \item[$\square$] User acceptance testing
\end{itemize}

\textbf{Tasks:}
\begin{itemize}
    \item \textbf{Yasser:} Y8 (Backend tests)
    \item \textbf{Zainab:} Z11 (Frontend tests)
    \item \textbf{Both:} S2-S6 (E2E, performance, security, UAT, bug fixes)
\end{itemize}

\subsubsection{July 2026: Final Deployment \& Submission}

\textbf{Goals:}
\begin{itemize}
    \item Production deployment
    \item Documentation complete
    \item FYP submitted
\end{itemize}

\textbf{Deliverables:}
\begin{itemize}
    \item[$\square$] Live production system
    \item[$\square$] Complete documentation
    \item[$\square$] User manual
    \item[$\square$] Final presentation ready
    \item[$\square$] FYP report submitted
\end{itemize}

\textbf{Tasks:}
\begin{itemize}
    \item \textbf{Both:} S7 (Documentation), S8 (Deployment), Final report
\end{itemize}

\subsection{Critical Milestones}

\begin{table}[h]
\centering
\begin{tabularx}{\textwidth}{|l|l|X|}
\hline
\rowcolor{codebg}
\textbf{Date} & \textbf{Milestone} & \textbf{Success Criteria} \\ \hline
Jan 31 & Setup Complete & Zainab can develop independently \\ \hline
Feb 28 & MVP Ready & Video calls + contact management working \\ \hline
Mar 31 & ML Integrated & Real sign recognition in calls \\ \hline
Apr 30 & Feature Complete & All major features implemented \\ \hline
May 31 & Polish Complete & Responsive, accessible, performant \\ \hline
Jun 30 & Testing Complete & 80\%+ coverage, bugs fixed \\ \hline
Jul 15 & Deployment & Live production system \\ \hline
Jul 31 & FYP Submission & Documentation \& report submitted \\ \hline
\end{tabularx}
\end{table}

% ============================================================
\section{Detailed Task Implementation Guides}
% ============================================================

\subsection{For Zainab: WebRTC Video Integration (Z1)}

\textbf{Priority:} \priority{Critical} \quad \textbf{Time Estimate:} 2 weeks

\subsubsection{Overview}

Implement peer-to-peer video calling using WebRTC and the existing SignalR hub.

\subsubsection{Prerequisites}
\begin{itemize}
    \item SignalR hub is fully implemented (CallHub.cs) ✅
    \item Frontend has CallSignalingClient service ✅
    \item simple-peer library can be used
\end{itemize}

\subsubsection{Step-by-Step Implementation}

\textbf{Step 1: Install Dependencies}

\begin{lstlisting}[language=bash]
cd ~/SilentTalkFYP/client
npm install simple-peer @types/simple-peer
\end{lstlisting}

\textbf{Step 2: Create WebRTC Service}

Create \texttt{src/services/webrtc.service.ts}:

\begin{lstlisting}[language=javascript, caption=webrtc.service.ts (partial)]
import Peer from 'simple-peer';

export class WebRTCService {
  private localStream: MediaStream | null = null;
  private peers: Map<string, Peer.Instance> = new Map();

  async getLocalStream(): Promise<MediaStream> {
    if (!this.localStream) {
      this.localStream = await navigator.mediaDevices
        .getUserMedia({ video: true, audio: true });
    }
    return this.localStream;
  }

  createPeer(userId: string, initiator: boolean): Peer.Instance {
    const peer = new Peer({ initiator, stream: this.localStream });
    this.peers.set(userId, peer);
    return peer;
  }

  // More methods for sending offers/answers
}
\end{lstlisting}

\textbf{Step 3: Update VideoCallPage Component}

Integrate WebRTC with SignalR:

\begin{lstlisting}[language=javascript, caption=VideoCallPage.tsx (partial)]
const VideoCallPage = () => {
  const [localStream, setLocalStream] = useState<MediaStream>();
  const [remoteStreams, setRemoteStreams] = useState<Map>();
  const signalingClient = useSignalRConnection();

  useEffect(() => {
    // Get local media
    webrtcService.getLocalStream().then(setLocalStream);

    // Listen for SignalR events
    signalingClient.on('userJoined', handleUserJoined);
    signalingClient.on('receiveOffer', handleOffer);
    signalingClient.on('receiveAnswer', handleAnswer);

    // Join call
    signalingClient.joinCall(callId);
  }, []);

  const handleUserJoined = (userId: string) => {
    // Create peer and send offer
    const peer = webrtcService.createPeer(userId, true);
    peer.on('signal', signal => {
      signalingClient.sendOffer(userId, signal);
    });
  };

  // More handlers...
};
\end{lstlisting}

\textbf{Step 4: Test Video Calling}

\begin{enumerate}
    \item Open two browser windows
    \item Login as different users
    \item Start a call
    \item Verify video/audio streaming
\end{enumerate}

\subsubsection{Success Criteria}
\begin{itemize}
    \item[$\square$] Local video displays in call
    \item[$\square$] Remote video displays from other user
    \item[$\square$] Audio works bidirectionally
    \item[$\square$] Can mute/unmute audio
    \item[$\square$] Can toggle video on/off
    \item[$\square$] Call ends gracefully
\end{itemize}

\subsubsection{Common Issues \& Solutions}

\begin{tcolorbox}[colback=warningorange!10, colframe=warningorange, title=Issue: Camera permission denied]
\textbf{Solution:} Add HTTPS in development or use localhost exception. Check browser console for permission errors.
\end{tcolorbox}

\begin{tcolorbox}[colback=warningorange!10, colframe=warningorange, title=Issue: SignalR disconnects]
\textbf{Solution:} Implement reconnection logic in CallSignalingClient. The backend hub already supports ReconnectToCall().
\end{tcolorbox}

\subsection{For Zainab: Call History Page (Z2)}

\textbf{Priority:} \priority{High} \quad \textbf{Time Estimate:} 3 days

\subsubsection{Backend API (Already Ready!)}

The CallController has these endpoints ready:

\begin{lstlisting}[language=bash]
GET /api/call/history?page=1&pageSize=20&status=Ended
GET /api/call/statistics
\end{lstlisting}

\subsubsection{Implementation Steps}

\textbf{Step 1: Create Call History Service}

\begin{lstlisting}[language=javascript, caption=src/services/callHistory.service.ts]
import axios from 'axios';

export interface CallHistoryItem {
  callId: string;
  initiatorName: string;
  startTime: string;
  endTime: string;
  duration: number; // minutes
  status: 'Ended' | 'Cancelled';
  recordingUrl?: string;
}

export const callHistoryService = {
  async getHistory(page = 1, pageSize = 20) {
    const response = await axios.get('/api/call/history', {
      params: { page, pageSize }
    });
    return response.data;
  },

  async getStatistics() {
    const response = await axios.get('/api/call/statistics');
    return response.data;
  }
};
\end{lstlisting}

\textbf{Step 2: Update CallHistoryPage Component}

\begin{lstlisting}[language=javascript, caption=src/pages/CallHistoryPage.tsx (partial)]
const CallHistoryPage = () => {
  const [calls, setCalls] = useState<CallHistoryItem[]>([]);
  const [stats, setStats] = useState(null);
  const [page, setPage] = useState(1);

  useEffect(() => {
    loadHistory();
    loadStats();
  }, [page]);

  const loadHistory = async () => {
    const data = await callHistoryService.getHistory(page);
    setCalls(data.items);
  };

  return (
    <div className="call-history">
      <h1>Call History</h1>
      <CallStatistics stats={stats} />
      <CallList calls={calls} />
      <Pagination page={page} onPageChange={setPage} />
    </div>
  );
};
\end{lstlisting}

\textbf{Step 3: Test}

\begin{enumerate}
    \item Make a few test calls
    \item End the calls
    \item Navigate to Call History page
    \item Verify calls are listed with correct details
\end{enumerate}

\subsubsection{Success Criteria}
\begin{itemize}
    \item[$\square$] Call history loads from backend
    \item[$\square$] Pagination works
    \item[$\square$] Statistics display (total calls, duration)
    \item[$\square$] Can click to view call details
    \item[$\square$] Recording download link works (if available)
\end{itemize}

\subsection{For Yasser: Contact Management Backend (Y1)}

\textbf{Priority:} \priority{High} \quad \textbf{Time Estimate:} 1 week

\subsubsection{Overview}

Implement backend API for contact/friend management. The Contact entity already exists.

\subsubsection{Implementation Steps}

\textbf{Step 1: Create ContactController}

Create \texttt{server/src/SilentTalk.Api/Controllers/ContactController.cs}:

\begin{lstlisting}[language=csharp, caption=ContactController.cs (partial)]
[ApiController]
[Route("api/[controller]")]
[Authorize]
public class ContactController : ControllerBase
{
    private readonly ApplicationDbContext _context;

    [HttpGet]
    public async Task<IActionResult> GetContacts()
    {
        var userId = User.GetUserId();
        var contacts = await _context.Contacts
            .Where(c => c.UserId == userId &&
                        c.Status == ContactStatus.Accepted)
            .Include(c => c.ContactUser)
            .ToListAsync();

        return Ok(contacts);
    }

    [HttpPost("request")]
    public async Task<IActionResult> SendContactRequest(
        [FromBody] ContactRequestDto dto)
    {
        // Create contact with Pending status
        // Send notification to other user
    }

    [HttpPost("{contactId}/accept")]
    public async Task<IActionResult> AcceptRequest(Guid contactId)
    {
        // Update status to Accepted
    }

    [HttpPost("{contactId}/block")]
    public async Task<IActionResult> BlockContact(Guid contactId)
    {
        // Update status to Blocked
    }
}
\end{lstlisting}

\textbf{Step 2: Add DTOs}

\begin{lstlisting}[language=csharp, caption=DTOs]
public class ContactRequestDto
{
    public string Email { get; set; } // or UserId
}

public class ContactDto
{
    public Guid ContactId { get; set; }
    public string DisplayName { get; set; }
    public string Email { get; set; }
    public string ProfileImageUrl { get; set; }
    public ContactStatus Status { get; set; }
    public bool IsOnline { get; set; }
}
\end{lstlisting}

\textbf{Step 3: Test Endpoints}

Use Swagger or curl:

\begin{lstlisting}[language=bash]
# Get contacts
curl -H "Authorization: Bearer <token>" \
  http://localhost:5000/api/contact

# Send request
curl -X POST \
  -H "Authorization: Bearer <token>" \
  -H "Content-Type: application/json" \
  -d '{"email":"friend@example.com"}' \
  http://localhost:5000/api/contact/request
\end{lstlisting}

\subsubsection{Success Criteria}
\begin{itemize}
    \item[$\square$] Can send contact request
    \item[$\square$] Can accept/reject requests
    \item[$\square$] Can block contacts
    \item[$\square$] Can get list of contacts
    \item[$\square$] Online status updated via SignalR
    \item[$\square$] Unit tests written
\end{itemize}

\subsection{For Both: End-to-End Testing (S2)}

\textbf{Priority:} \priority{High} \quad \textbf{Time Estimate:} 2 weeks

\subsubsection{Tools}
\begin{itemize}
    \item \textbf{Frontend:} Playwright or Cypress
    \item \textbf{Backend:} xUnit integration tests
\end{itemize}

\subsubsection{Critical User Flows to Test}

\begin{enumerate}
    \item \textbf{Authentication Flow}
    \begin{itemize}
        \item Register → Email verification → Login → Access protected page
    \end{itemize}

    \item \textbf{Video Call Flow}
    \begin{itemize}
        \item Login → Schedule call → Join call → Video/audio working → End call
    \end{itemize}

    \item \textbf{Contact Management Flow}
    \begin{itemize}
        \item Login → Add contact → Accept request → Start call with contact
    \end{itemize}
\end{enumerate}

\subsubsection{Implementation Example (Playwright)}

\begin{lstlisting}[language=javascript, caption=tests/e2e/auth.spec.ts]
import { test, expect } from '@playwright/test';

test('user can register and login', async ({ page }) => {
  // Register
  await page.goto('http://localhost:3000/register');
  await page.fill('[name="email"]', 'test@example.com');
  await page.fill('[name="password"]', 'Test123!');
  await page.fill('[name="confirmPassword"]', 'Test123!');
  await page.fill('[name="displayName"]', 'Test User');
  await page.click('button[type="submit"]');

  // Verify redirect to login
  await expect(page).toHaveURL('/login');

  // Login
  await page.fill('[name="email"]', 'test@example.com');
  await page.fill('[name="password"]', 'Test123!');
  await page.click('button[type="submit"]');

  // Verify redirect to home
  await expect(page).toHaveURL('/');
  await expect(page.locator('text=Test User')).toBeVisible();
});
\end{lstlisting}

% ============================================================
\section{Collaboration Workflow}
% ============================================================

\subsection{Weekly Meeting Structure}

\subsubsection{Monday Planning Meeting (30 minutes)}

\textbf{Agenda:}
\begin{enumerate}
    \item Review previous week's accomplishments
    \item Discuss blockers and dependencies
    \item Plan current week's tasks
    \item Assign priorities
\end{enumerate}

\textbf{Format:}
\begin{itemize}
    \item Each person shares: What I did, what I'm doing, blockers
    \item Update task board (Trello/Notion/GitHub Projects)
    \item Set 3-5 key goals for the week
\end{itemize}

\subsubsection{Wednesday Quick Sync (15 minutes)}

\textbf{Agenda:}
\begin{itemize}
    \item Progress check
    \item Any blockers?
    \item Any help needed?
\end{itemize}

\subsubsection{Friday Review (30 minutes)}

\textbf{Agenda:}
\begin{enumerate}
    \item Demo completed features
    \item Review code quality
    \item Discuss learnings
    \item Preview next week
\end{enumerate}

\subsection{Git Workflow}

\subsubsection{Branch Naming Convention}

\begin{lstlisting}[language=bash]
# Feature branches
feature/z1-webrtc-integration
feature/y1-contact-backend

# Bug fixes
fix/login-validation-error
fix/video-stream-freeze

# Improvements
improve/responsive-navbar
improve/api-performance
\end{lstlisting}

\subsubsection{Development Workflow}

\textbf{Step 1: Create Feature Branch}
\begin{lstlisting}[language=bash]
# Always branch from main
git checkout main
git pull origin main

# Create feature branch
git checkout -b feature/z2-call-history
\end{lstlisting}

\textbf{Step 2: Develop \& Commit}
\begin{lstlisting}[language=bash]
# Make changes
# ...

# Stage and commit
git add .
git commit -m "feat: add call history page with pagination"

# Follow conventional commits:
# feat: new feature
# fix: bug fix
# docs: documentation
# style: formatting
# refactor: code restructuring
# test: adding tests
# chore: maintenance
\end{lstlisting}

\textbf{Step 3: Push \& Create PR}
\begin{lstlisting}[language=bash]
# Push to remote
git push origin feature/z2-call-history

# Create Pull Request on GitHub
# - Add description
# - Link related issues
# - Request review from partner
\end{lstlisting}

\textbf{Step 4: Code Review}
\begin{itemize}
    \item Partner reviews within 24 hours
    \item Address feedback
    \item Merge when approved
\end{itemize}

\subsubsection{Pull Request Template}

\begin{lstlisting}[language=markdown]
## Description
Brief description of changes

## Type of Change
- [ ] New feature
- [ ] Bug fix
- [ ] Breaking change
- [ ] Documentation update

## Testing Done
- [ ] Unit tests added/updated
- [ ] Manually tested locally
- [ ] Tested with other features

## Screenshots (if UI change)
[Add screenshots]

## Checklist
- [ ] Code follows style guidelines
- [ ] Self-review completed
- [ ] Comments added for complex code
- [ ] No console.log() left in code
- [ ] Documentation updated
\end{lstlisting}

\subsection{Code Review Guidelines}

\subsubsection{What to Look For}

\begin{enumerate}
    \item \textbf{Functionality:} Does it work as intended?
    \item \textbf{Code Quality:} Clean, readable, maintainable?
    \item \textbf{Best Practices:} Follows conventions?
    \item \textbf{Performance:} Any optimization issues?
    \item \textbf{Security:} No vulnerabilities?
    \item \textbf{Testing:} Adequate test coverage?
\end{enumerate}

\subsubsection{Review Comments Guidelines}

\begin{itemize}
    \item Be constructive, not critical
    \item Explain the "why" behind suggestions
    \item Use prefixes:
    \begin{itemize}
        \item \textit{NIT:} Minor nitpick (optional)
        \item \textit{Q:} Question for clarification
        \item \textit{BLOCKER:} Must fix before merge
        \item \textit{SUGGESTION:} Consider this approach
    \end{itemize}
    \item Praise good code!
\end{itemize}

\subsection{Communication Best Practices}

\subsubsection{Daily Updates (Async)}

Use WhatsApp/Slack for quick updates:

\begin{tcolorbox}[colback=secondarygreen!10, colframe=secondarygreen, title=Example Daily Update]
\textbf{Today's Progress:}\\
✅ Completed call history page\\
✅ Connected to backend API\\
🚧 Working on pagination component\\
\\
\textbf{Tomorrow:}\\
🎯 Finish pagination\\
🎯 Start profile page integration\\
\\
\textbf{Blockers:}\\
None
\end{tcolorbox}

\subsubsection{Asking for Help}

When stuck, provide context:

\begin{tcolorbox}[colback=warningorange!10, colframe=warningorange, title=Good Help Request]
\textbf{Problem:} SignalR connection keeps dropping during video calls\\
\\
\textbf{What I tried:}
\begin{itemize}
    \item Checked network tab - seeing 101 switching protocols
    \item Added reconnection logic
    \item Still disconnects after ~2 minutes
\end{itemize}
\textbf{Error:} [paste error message]\\
\\
\textbf{Question:} Is there a timeout setting in the backend hub?
\end{tcolorbox}

\subsection{Task Management}

\subsubsection{Recommended Tool: GitHub Projects}

Create a project board with columns:
\begin{itemize}
    \item \textbf{Backlog} - All tasks
    \item \textbf{Todo} - Planned for current sprint
    \item \textbf{In Progress} - Currently working on
    \item \textbf{Review} - Awaiting code review
    \item \textbf{Done} - Completed and merged
\end{itemize}

\subsubsection{Sprint Structure (2-week sprints)}

\begin{enumerate}
    \item \textbf{Sprint Planning} (Monday Week 1)
    \begin{itemize}
        \item Select tasks from backlog
        \item Assign to team members
        \item Estimate time
    \end{itemize}

    \item \textbf{Daily Work} (async updates)
    \item \textbf{Mid-Sprint Check} (Wednesday Week 1)
    \item \textbf{Sprint Review} (Friday Week 2)
    \begin{itemize}
        \item Demo completed features
        \item Move unfinished tasks to next sprint
    \end{itemize}
\end{enumerate}

% ============================================================
\section{Technical Reference}
% ============================================================

\subsection{Backend API Endpoints}

\subsubsection{Authentication Endpoints}

\begin{longtable}{|p{2cm}|p{5cm}|p{5cm}|}
\hline
\rowcolor{codebg}
\textbf{Method} & \textbf{Endpoint} & \textbf{Description} \\ \hline
\endfirsthead
\hline
\rowcolor{codebg}
\textbf{Method} & \textbf{Endpoint} & \textbf{Description} \\ \hline
\endhead

POST & /api/auth/register & Register new user \\ \hline
POST & /api/auth/login & Login and get JWT token \\ \hline
POST & /api/auth/logout & Invalidate refresh token \\ \hline
POST & /api/auth/refresh & Refresh access token \\ \hline
GET & /api/auth/verify-email & Verify email with token \\ \hline
POST & /api/auth/forgot-password & Request password reset \\ \hline
POST & /api/auth/reset-password & Reset password with token \\ \hline
GET & /api/auth/me & Get current user info \\ \hline
\caption{Auth Endpoints}
\end{longtable}

\subsubsection{Call Management Endpoints}

\begin{longtable}{|p{2cm}|p{5cm}|p{5cm}|}
\hline
\rowcolor{codebg}
\textbf{Method} & \textbf{Endpoint} & \textbf{Description} \\ \hline
\endfirsthead
\hline
\rowcolor{codebg}
\textbf{Method} & \textbf{Endpoint} & \textbf{Description} \\ \hline
\endhead

POST & /api/call/schedule & Schedule a call \\ \hline
POST & /api/call/start & Start instant call \\ \hline
GET & /api/call/\{id\} & Get call details \\ \hline
POST & /api/call/\{id\}/end & End call \\ \hline
GET & /api/call/history & Get call history (paginated) \\ \hline
GET & /api/call/statistics & Get user call statistics \\ \hline
POST & /api/call/\{id\}/recording & Upload recording \\ \hline
GET & /api/call/\{id\}/recording & Download recording URL \\ \hline
\caption{Call Endpoints}
\end{longtable}

\subsection{SignalR Hub Methods}

\subsubsection{CallHub Events}

\begin{longtable}{|p{4cm}|p{8cm}|}
\hline
\rowcolor{codebg}
\textbf{Method} & \textbf{Description} \\ \hline
\endfirsthead
\hline
\rowcolor{codebg}
\textbf{Method} & \textbf{Description} \\ \hline
\endhead

JoinCall(callId) & Join a call room \\ \hline
LeaveCall(callId) & Leave call room \\ \hline
SendOffer(userId, offer) & Send WebRTC offer \\ \hline
SendAnswer(userId, answer) & Send WebRTC answer \\ \hline
SendIceCandidate(userId, candidate) & Send ICE candidate \\ \hline
UpdateMediaState(audio, video) & Toggle audio/video \\ \hline
SendChatMessage(callId, message) & Send in-call chat \\ \hline
StartScreenshare() & Start screen sharing \\ \hline
StartRecording() & Start call recording \\ \hline
\caption{SignalR Hub Methods}
\end{longtable}

\subsection{Essential Commands Reference}

\subsubsection{Docker Commands}

\begin{lstlisting}[language=bash]
# Start application
cd ~/SilentTalkFYP
./start.sh

# Stop application
./stop.sh

# View logs
cd infrastructure/docker
docker-compose logs -f server      # Backend
docker-compose logs -f client      # Frontend
docker-compose logs -f ml-service  # ML service

# Restart specific service
docker-compose restart server

# Rebuild after code changes
docker-compose up -d --build server

# Check running containers
docker ps

# Check service health
curl http://localhost:5000/health
curl http://localhost:8000/status
\end{lstlisting}

\subsubsection{Frontend Development Commands}

\begin{lstlisting}[language=bash]
cd ~/SilentTalkFYP/client

# Install dependencies
npm install

# Start dev server (if not using Docker)
npm run dev

# Build for production
npm run build

# Run tests
npm test

# Run linter
npm run lint

# Format code
npm run format
\end{lstlisting}

\subsubsection{Backend Development Commands}

\begin{lstlisting}[language=bash]
cd ~/SilentTalkFYP/server

# Build solution
dotnet build

# Run API (if not using Docker)
dotnet run --project src/SilentTalk.Api

# Run tests
dotnet test

# Create migration
dotnet ef migrations add MigrationName \
  --project src/SilentTalk.Infrastructure

# Update database
dotnet ef database update \
  --project src/SilentTalk.Api
\end{lstlisting}

\subsection{Common Issues \& Solutions}

\subsubsection{Frontend Issues}

\begin{tcolorbox}[colback=dangerred!10, colframe=dangerred, title=Issue: CORS error when calling API]
\textbf{Symptom:} Browser console shows "CORS policy blocked"\\
\\
\textbf{Solution:}
\begin{enumerate}
    \item Verify backend CORS is configured in Program.cs
    \item Check VITE\_API\_URL is correct (\url{http://localhost:5000})
    \item Ensure backend container is running
    \item Check Network tab for actual URL being called
\end{enumerate}
\end{tcolorbox}

\begin{tcolorbox}[colback=dangerred!10, colframe=dangerred, title=Issue: SignalR connection fails]
\textbf{Symptom:} "Failed to start the connection"\\
\\
\textbf{Solution:}
\begin{itemize}
    \item Check VITE\_WS\_URL is correct (ws://localhost:5000)
    \item Verify backend SignalR hub is running
    \item Check browser console for detailed error
    \item Ensure JWT token is included in connection
\end{itemize}
\end{tcolorbox}

\subsubsection{Backend Issues}

\begin{tcolorbox}[colback=dangerred!10, colframe=dangerred, title=Issue: Database connection fails]
\textbf{Symptom:} API throws connection timeout\\
\\
\textbf{Solution:}
\begin{enumerate}
    \item Check PostgreSQL container is running: \texttt{docker ps}
    \item Verify connection string in appsettings.json
    \item Check database is initialized: \texttt{docker exec -it silents-talk-postgres psql -U silentstalk}
    \item Run migrations: \texttt{dotnet ef database update}
\end{enumerate}
\end{tcolorbox}

\subsection{Performance Optimization Tips}

\subsubsection{Frontend Optimization}

\begin{enumerate}
    \item \textbf{Code Splitting}
    \begin{lstlisting}[language=javascript]
// Lazy load pages
const VideoCallPage = lazy(() =>
  import('./pages/VideoCallPage')
);
    \end{lstlisting}

    \item \textbf{Memoization}
    \begin{lstlisting}[language=javascript]
// Prevent unnecessary re-renders
const MemoizedComponent = React.memo(MyComponent);
    \end{lstlisting}

    \item \textbf{Image Optimization}
    \begin{itemize}
        \item Use WebP format
        \item Lazy load images
        \item Use appropriate sizes
    \end{itemize}
\end{enumerate}

\subsubsection{Backend Optimization}

\begin{enumerate}
    \item \textbf{Database Queries}
    \begin{itemize}
        \item Use Select() to project only needed fields
        \item Add indexes on frequently queried columns
        \item Use AsNoTracking() for read-only queries
    \end{itemize}

    \item \textbf{Caching}
    \begin{lstlisting}[language=csharp]
// Cache frequently accessed data
var cachedData = await _cache.GetOrCreateAsync(
    key: "user:profile:123",
    factory: async () => await GetUserProfile(123),
    expiration: TimeSpan.FromMinutes(5)
);
    \end{lstlisting}
\end{enumerate}

% ============================================================
\section{Quality Assurance Checklist}
% ============================================================

\subsection{Before Each Pull Request}

\begin{itemize}
    \item[$\square$] Code builds without errors
    \item[$\square$] All tests pass
    \item[$\square$] No console.log() or debugger statements
    \item[$\square$] Code follows style guide
    \item[$\square$] Comments added for complex logic
    \item[$\square$] No hardcoded credentials or secrets
    \item[$\square$] Responsive design tested (mobile/tablet/desktop)
    \item[$\square$] Browser compatibility checked
    \item[$\square$] Accessibility checked (keyboard navigation, screen reader)
\end{itemize}

\subsection{Before Each Demo/Milestone}

\begin{itemize}
    \item[$\square$] All features working end-to-end
    \item[$\square$] No critical bugs
    \item[$\square$] Performance acceptable (<2s page load)
    \item[$\square$] Data persists across restarts
    \item[$\square$] Error handling graceful
    \item[$\square$] Documentation updated
    \item[$\square$] Demo script prepared
\end{itemize}

\subsection{Final Submission Checklist}

\begin{itemize}
    \item[$\square$] All features from requirements implemented
    \item[$\square$] ML model accuracy ≥85\%
    \item[$\square$] Video calling works reliably
    \item[$\square$] Accessibility compliance (WCAG 2.1 AA)
    \item[$\square$] Security audit passed
    \item[$\square$] Performance benchmarks met
    \item[$\square$] Code test coverage ≥80\%
    \item[$\square$] User documentation complete
    \item[$\square$] Admin documentation complete
    \item[$\square$] Deployment guide written
    \item[$\square$] Final report submitted
\end{itemize}

% ============================================================
\section{Conclusion \& Final Notes}
% ============================================================

\subsection{Keys to Success}

\begin{enumerate}
    \item \textbf{Communication is Critical}
    \begin{itemize}
        \item Daily async updates
        \item Weekly sync meetings
        \item Ask questions early
        \item Share blockers immediately
    \end{itemize}

    \item \textbf{Stay Organized}
    \begin{itemize}
        \item Use task board (GitHub Projects)
        \item Follow Git workflow
        \item Document decisions
        \item Track time estimates
    \end{itemize}

    \item \textbf{Quality Over Speed}
    \begin{itemize}
        \item Write tests
        \item Review code thoroughly
        \item Refactor when needed
        \item Don't skip documentation
    \end{itemize}

    \item \textbf{Support Each Other}
    \begin{itemize}
        \item Help when partner is stuck
        \item Share learnings
        \item Celebrate small wins
        \item Be patient and respectful
    \end{itemize}
\end{enumerate}

\subsection{Emergency Contacts}

\begin{table}[h]
\centering
\begin{tabularx}{\textwidth}{|l|X|}
\hline
\rowcolor{codebg}
\textbf{Person} & \textbf{Contact} \\ \hline
Yasser & [Email/Phone] \\ \hline
Zainab & [Email/Phone] \\ \hline
FYP Supervisor & [Email] \\ \hline
\end{tabularx}
\end{table}

\subsection{Resources}

\begin{itemize}
    \item \textbf{Documentation Folder:} \texttt{/docs} in repository
    \item \textbf{Project Status:} \texttt{PROJECT\_STATUS.md}
    \item \textbf{API Docs:} \url{http://localhost:5000/docs} (when running)
    \item \textbf{Quick Start:} \texttt{QUICK\_START.md}
\end{itemize}

\subsection{Acknowledgments}

This project represents months of hard work and dedication. By following this collaboration guide, maintaining clear communication, and supporting each other, you will successfully deliver a high-quality Final Year Project that makes a real difference in the deaf and hard-of-hearing community.

\vspace{1cm}

\textit{Good luck, and happy coding! 🚀}

\vspace{1cm}

\begin{center}
\textbf{Version 1.0 - January 2026}\\
\textit{Last Updated: \today}
\end{center}

\end{document}
